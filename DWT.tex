\documentclass[a4,12pt]{scrartcl}
\usepackage{ngerman}
\usepackage[utf8]{inputenc}
\usepackage{amsmath}
\usepackage{hyperref}
\usepackage{graphicx}
\usepackage{amssymb}
\usepackage{multirow}

\title{Diskrete Wahrscheinlichkeitstheorie}
\subtitle{Zusammenfassung für die Klausurvorbereitung}
\author{Christian Rupp}
\date{\today}

\begin{document}

\maketitle

\newpage

\tableofcontents

\section{Vorwort}

Dieses Dokument orientiert sich an den Inhalten der Vorlesung Diskrete Wahrscheinlichkeitstheorie der Fakultät für Informatik der Technischen Universität München aus dem Sommersemester 2014.
Es erhebt keinen Anspruch auf Vollständigkeit und Korrektheit.

\section{Hilfreiche Formeln}

\begin{itemize}
\item Allgemeine Binomische Formel: $(a+b)^n=\sum^n_{i=0}\binom {n}{i}a^nb^{n-i}$
\item $\sum^r_{x=0}\binom{a}{x}\binom{b}{r-x}=\binom{a+b}{r}$
\item $\left(1+\frac{1}{n}\right)^n=e$
\end{itemize}

\subsection{Kombinatorik}
Anzahl der Verteilungen von v Bällen auf m Urnen.\\
\begin{tabular}{|c||c|c|c|c|}
\hline
&beliebig &höchstens &mindestens&genau\\
&viele Bälle&ein Ball& ein Ball& ein Ball\\
& pro Urne& pro Urne& pro Urne& pro Urne\\
& (beliebig)& (injektiv) & (surjektiv)& (bijektiv)\\
\hline
\hline
Bälle unterscheidbar& \multirow{2}{*}{$m^n$} &\multirow{2}{*}{$m^\underline{n}$} &\multirow{2}{*}{$m!*S_{n,m}$} &\multirow{2}{*}{$n!$}\\
Urnen unterscheidbar &&&&\\
\hline
Bälle gleich& \multirow{2}{*}{$\binom{n+m-1}{n}$} &\multirow{2}{*}{$\binom{m}{n}$} &\multirow{2}{*}{$\binom{n-1}{m-1}$} &\multirow{2}{*}{$1$}\\
Urnen unterscheidbar &&&&\\
\hline
Bälle unterscheidbar& \multirow{2}{*}{$\sum^m_{k=0}S_{n,k}$} &\multirow{2}{*}{$1$} &\multirow{2}{*}{$S_{n,m}$} &\multirow{2}{*}{$1$}\\
Urnen gleich &&&&\\
\hline
Bälle gleich& \multirow{2}{*}{$\sum^m_{k=0}P_{n,k}$} &\multirow{2}{*}{$1$} &\multirow{2}{*}{$P_{n,m}$} &\multirow{2}{*}{$1$}\\
Urnen gleich &&&&\\
\hline
\end{tabular}

\section{Diskrete Wahrscheinlichkeitsräume}

\subsection{Grundlagen}

\begin{itemize}
\item diskreter Wahrscheinlichkeitsraum: $\Omega = \{\omega_1,\ldots,\omega_n\}$ $|$ $n\in \mathbb{N}$
\item Elementarereignis:\\
	\begin{itemize}
	\item $0 \leq \Pr[\omega_i]\leq 1$
	\item $\sum_{i=1}^n\Pr[\omega_i]=1$
	\item $\Pr[\omega_i]:=\frac{1}{|\Omega|}$
	\end{itemize}
\item Ereignis:\\
	\begin{itemize}
	\item $E\subseteq\Omega$
	\item $\Pr[E]:=\sum_{\omega\in E}\Pr[\omega]$
	\item $\Pr[E]:=\frac{|E|}{|\Omega|}$
	\end{itemize}
\item $\Pr[\emptyset]=0$,$\Pr[\Omega]=1$ 
\item $0\leq \Pr[A] \leq 1$
\item $\Pr[\bar A]=1-\Pr[A]$
\item $A\subseteq B \Rightarrow \Pr[A]\leq \Pr[B]$
\end{itemize}

Laplace verteilt heißt, das jedes Elementarereignis gleich wahrscheinlich ist.

\subsubsection{disjunkte Ereignisse}
$\forall(i,j)\in\mathbb{N}: i\neq j, A_i\cap  A_j$

\end{document}