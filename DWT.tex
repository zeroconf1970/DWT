\documentclass[a4,12pt]{scrartcl}
\usepackage{ngerman}
\usepackage[utf8]{inputenc}
\usepackage{amsmath}
\usepackage{hyperref}
\usepackage{graphicx}
\usepackage{amssymb}

\title{Diskrete Wahrscheinlichkeitstheorie}
\subtitle{Zusammenfassung für die Klausurvorbereitung}
\author{Christian Rupp}
\date{\today}

\begin{document}

\maketitle

\newpage

\tableofcontents

\section{Vorwort}

Dieses Dokument orientiert sich an den Inhalten der Vorlesung Diskrete Wahrscheinlichkeitstheorie der Fakultät für Informatik der Technischen Universität München aus dem Sommersemester 2014.
Es erhebt keinen Anspruch auf Vollständigkeit und Korrektheit.

\section{Hilfreiche Formeln}

\begin{itemize}
\item Allgemeine Binomische Formel: $(a+b)^n=\sum^n_{i=0}\binom {n}{i}a^nb^{n-i}$
\item $\sum^r_{x=0}\binom{a}{x}\binom{b}{r-x}=\binom{a+b}{r}$
\end{itemize}

\section{Diskrete Wahrscheinlichkeitsräume}

\subsection{Grundlagen}

\begin{itemize}
\item diskreter Wahrscheinlichkeitsraum: $\Omega = \{\omega_1,\ldots,\omega_n\}$ $|$ $n\in \mathbb{N}$
\item Elementarereignis:\\
	\begin{itemize}
	\item $0 \leq Pr[\omega_i]\leq 1$
	\item $\sum_{i=1}^nPr[\omega_i]=1$
	\item $Pr[\omega_i]:=\frac{1}{|\Omega|}$
	\end{itemize}
\item Ereignis:\\
	\begin{itemize}
	\item $E\subseteq\Omega$
	\item $Pr[E]:=\sum_{\omega\in E}Pr[\omega]$
	\item $Pr[E]:=\frac{|E|}{|\Omega|}$
	\end{itemize}
\item $Pr[\emptyset]=0$,$Pr[\Omega]=1$ 
\item $0\leq Pr[A] \leq 1$
\item $Pr[\overline{A}]=1-Pr[A]$
\item $A\subseteq B \Rightarrow Pr[A]\leq Pr[B]$
\end{itemize}

\end{document}