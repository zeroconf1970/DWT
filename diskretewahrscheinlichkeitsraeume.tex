
\section{Diskrete Wahrscheinlichkeitsräume}

\subsection{Grundlagen}

\begin{itemize}
\item diskreter Wahrscheinlichkeitsraum: $\Omega = \{\omega_1,\ldots,\omega_n\}$ $|$ $n\in \mathbb{N}$
\item Elementarereignis:\\
	\begin{itemize}
	\item $0 \leq \Pr[\omega_i]\leq 1$
	\item $\sum_{i=1}^n\Pr[\omega_i]=1$
	\item $\Pr[\omega_i]:=\frac{1}{|\Omega|}$
	\end{itemize}
\item Ereignis:\\
	\begin{itemize}
	\item $E\subseteq\Omega$
	\item $\Pr[E]:=\sum_{\omega\in E}\Pr[\omega]$
	\item $\Pr[E]:=\frac{|E|}{|\Omega|}$
	\end{itemize}
\item $\Pr[\emptyset]=0$,$\Pr[\Omega]=1$ 
\item $0\leq \Pr[A] \leq 1$
\item $\Pr[\bar A]=1-\Pr[A]$
\item $A\subseteq B \Rightarrow \Pr[A]\leq \Pr[B]$
\end{itemize}

Laplace verteilt heißt, das jedes Elementarereignis gleich wahrscheinlich ist.

\subsubsection{disjunkte Ereignisse}
$\forall(i,j)\in\mathbb{N}: i\neq j, A_i\cap  A_j$