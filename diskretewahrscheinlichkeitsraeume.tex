
\section{Diskrete Wahrscheinlichkeitsräume}

\subsection{Grundlagen}

\begin{itemize}
\item diskreter Wahrscheinlichkeitsraum: $\Omega = \{\omega_1,\ldots,\omega_n\}$ $|$ $n\in \mathbb{N}$
\item Elementarereignis:\\
	\begin{itemize}
	\item $0 \leq \Pr[\omega_i]\leq 1$
	\item $\sum_{i=1}^n\Pr[\omega_i]=1$
	\item $\Pr[\omega_i]:=\frac{1}{|\Omega|}$
	\end{itemize}
\item Ereignis:\\
	\begin{itemize}
	\item $E\subseteq\Omega$
	\item $\Pr[E]:=\sum_{\omega\in E}\Pr[\omega]$
	\item $\Pr[E]:=\frac{|E|}{|\Omega|}$
	\end{itemize}
\item $\Pr[\emptyset]=0$,$\Pr[\Omega]=1$ 
\item $0\leq \Pr[A] \leq 1$
\item $\Pr[\bar A]=1-\Pr[A]$
\item $A\subseteq B \Rightarrow \Pr[A]\leq \Pr[B]$
\end{itemize}

Laplace verteilt heißt, das jedes Elementarereignis gleich wahrscheinlich ist.

\subsubsection{disjunkte Ereignisse}
$\forall(i,j)\in\mathbb{N}: i\neq j, A_i\cap  A_j=\emptyset$
\begin{itemize}
\item $\Pr[\cup^n_{i=1}A_i]=\sum^n_{i=1}\Pr[A_i]$
\item $\Pr[A\cup B]=\Pr[A]+\Pr[B]$
\item $\Pr[\cup^\infty_{i=1}]=\sum^\infty_{i=1}\Pr[A_i]$
\end{itemize}

\subsubsection{Siebformel,Prinzip der Inklusion/Exklusion}
\begin{itemize}
\item Zwei Ereignisse: $\Pr[A\cup B]=\Pr[A]+\Pr[B]-\Pr[A\cap B]$
\item Drei Ereignisse: $\Pr[A_1\cup A_2\cup A_3]=\Pr[A_1]+\Pr[A_2]+\Pr[A_3]\ldots$\\
				$-\Pr[A_1\cap A_2]-\Pr[A_1\cap A_3]-\Pr[A_2\cap A_3]\ldots$\\
				$+\Pr[A_1\cap A_2\cap A_3]$
\item Allgemeiner Fall: $\Pr[\cup^n_{i=1}A_i]=\sum_{i=1}^n\Pr[A_i]-\sum_{1\leq i_1<i_2\leq n}\Pr[A_{i_1}\cap A_{i_2}]\pm \ldots$\\
				$+(-1)^{l-1}\sum_{1\leq i_1<\ldots<i_2\leq n}\Pr[A_{i_1}\cap\ldots\cap A_{i_2}]\pm\ldots$\\
				$+(-1)^{l-1}\Pr[A_1\cap\ldots\cap A_n]$
\end{itemize}
\\
\underline{Anmerkung}: Üblicherweise benötigt man den Allgemeinen Fall während dieser Vorlesung nicht!\\

\subsubsection{Boolsche Ungleichung}

\begin{itemize}
\item Ereignisse $A_1,\ldots, A_n$: $\Pr[\cup_{i=1}^n A_i]\leq\sum_{i=1}^n\Pr[A_i]$
\item unendliche Ereignisse:$\Pr[\cup^\infty_{i=1} A_i]\leq\sum^\infty_{i=1}\Pr[A_i]$
\end{itemize}
\\
\underline{Anmerkung}: Hiermit kann man den Allgemeinen Fall der Siebformel abschätzen.

\subsubsection{Wahl der Wahrscheinlichkeiten}
Sofern nichts anderes gegeben ist, gilt das Prinzip von Laplace und alle Elementarereignisse sind gleichwahrscheinlich.\\
$\Pr[E]=\frac{|E|}{|\Omega |}$

\subsection{Bedingte Wahrscheinlichkeiten}
\begin{itemize}
\item $\Pr[B|B]=1$
\item $\Pr[A|\Omega]=\Pr[A]$
\item $\Pr[A|B]:=\frac{\Pr[A\cap B]}{\Pr[B]}$
\end{itemize}
\\
\underline{Anmerkung}: $\Pr[\emptyset | B]= 0$ und $\Pr[\bar A|B]=1-\Pr[A|B]$\\
\underline{Andere Schreibweise}: $\Pr[A\cap B]=\Pr[B|A]*\Pr[A]= \Pr[A|B]*\Pr[B]$

\subsubsection{Multiplikationssatz}
$\Pr[A_1\cap\ldots\cap A_n]>0\Rightarrow\Pr[A_1]*\Pr[A_2|A_1]*\Pr[A_3|A_1\cap A_2]*\ldots$\\
$\dots*\Pr[A_n|A_1\cap\ldots\cap A_{n-1}]$

\subsubsection{Satz der totalen Wahrscheinlichkeit}
Es gilt das alle Ereignisse paarweise disjunkt sind und B die Vereinigung dieser Ereignisse ist!
\begin{itemize}
\item endlicher Fall: $\Pr[B]=\sum_{i=1}^n\Pr[B|A_i]*\Pr[A_i]$
\item unendlicher Fall: $\Pr[B]=\sum_{i=1}^\infty\Pr[B|A_i]*\Pr[A_i]$
\end{itemize}
\\
\underline{Nützliche Erkenntnisse}:Seien $B\subseteq A\cup\bar A$ und $A\cap\bar A =\emptyset$ dann gilt $\Pr[B]=\Pr[B|A]*\Pr[A]+\Pr[B|\bar A]*\Pr[\bar A]$

\subsubsection{Satz von Bayes}
Es gilt alle Ereignisse sind paarweise disjunkt und größer Null und B die Vereinigung dieser Ereignisse und größer Null.
\begin{itemize}
\item Dann gilt für $i=1,\ldots,n$: $\Pr[A_i|B]=\frac{\Pr[A_i\cap B]}{\Pr[B]}=\frac{\Pr[B|A_i]*\Pr[A_i]}{\sum^n_{j=1}\Pr[B|A_j]*\Pr[A_j]}$
\item Dann gilt für $i=1,\ldots$: $\Pr[A_i|B]=\frac{\Pr[A_i\cap B]}{\Pr[B]}=\frac{\Pr[B|A_i]*\Pr[A_i]}{\sum^\infty_{j=1}\Pr[B|A_j]*\Pr[A_j]}$
\end{itemize}

\subsection{Unabhängigkeit}
Unabhängigkeit heißt das A und B sich nicht gegenseitig beeinflussen, d.h. es gilt $\Pr[A|B]=\Pr[A]$ oder $\Pr[B|A]=\Pr[B]$.

\begin{itemize}
\item $\Pr[A\cap B]=\Pr[A]*\Pr[B]$
\item Falls $\Pr[B]\neq0$: $\Pr[A]=\frac{\Pr[A\cap B]}{\Pr[B]}=\Pr[A|B]$
\item $A_1,\ldots,A_n$ heißen unabhängig, wenn für alle Teilmengen $I = \{ i_1,\ldots, i_k\}\subseteq\{1,\ldots,n\}$ mit $i_1<\ldots<i_k$ gilt:\\
	$\Pr[A_{i_1}\cap\ldots\cap A_{i_k}]=\Pr[A_{i_1}]*\ldots*\Pr[A_{i_k}]$
\item $A_1,\ldots,A_n$ genau dann unabhängig, wenn $\forall (s_1,\ldots,s_n)\in\{0,1\}^n$ gilt, dass\\
	$\Pr[A_1^{s_1}\cap\ldots\cap A_n^{s_n}]=\Pr[A_1^{s_1}]*\ldots*\Pr[A_1^{s_n}]$ mit $A_i^0=\bar A_i$ und $A_i^1=A_i$.
\end{itemize}

\subsection{Zufallsvariablen}

\subsubsection{Grundlagen}
\begin{itemize}
\item $X:\Omega\rightarrow\mathbb{R}$
\item $W_X:=X(\Omega)=\{x\in\mathbb{R};\exists\omega\in\Omega\text{ mit } X(\omega)=x\}$
\item $\Pr[X\leq x_i]=\sum_{x\in W_x:x\leq x_i}\Pr[X=x]=\Pr[\{\omega\in\Omega ;X(\omega)\leq x_i\}]$
\item (diskrete) Dichte(funktion): $f_{X}:\mathbb{R}\ni x\mapsto\Pr[X=x]\in[0,1]$
\item Verteilung(sfunktion): $F_X:\mathbb{R}\ni x\mapsto\Pr[X\leq x]=\sum_{x'\in W_x:x'\leq x}\Pr[X=x']\in[0,1]$
\end{itemize}

\subsubsection{Erwartungswert und Varianz}
\begin{itemize}
\item Sofern $\sum_{x\in W_x}|x|*\Pr[X=x]$ konvergiert gilt:\\
	$\mathbb{E}:=\sum_{x\in W_x}x*\Pr[X=x]=\sum_{x\in W_x}x*f_X(x)$
\item Monotonie des Erwartungswertes: $X(\omega)\leq Y(\omega)\forall\omega\in\Omega\Rightarrow \mathbb{E}[X]\leq\mathbb{E}[Y]$
\item $\mathbb{E}[a*X+b]=a*\mathbb{E}[X]+b$
\item $\mathbb{E}[X]=\sum_{i=1}^\infty\Pr[X\geq i]=\sum_{i=0}^\infty i*\Pr[X=i]$
\end{itemize}


